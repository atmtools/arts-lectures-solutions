\documentclass[paper=a4, fontsize=11pt]{scrartcl} % A4 paper and 11pt font size
\input{../common/header.tex}

\title{Exercise 4: Atmospheric Brightness Temperature Spectra}
\author{Sample Solution}
\date{Effective: 06.02.2019}

%%%%%%%%%%%%%%%%%%%%%%%%%%%%%%%%%%%%%%%%%%%%%%%%%%%%%%%%%%%%%%%%%%%%%%%%%%%%%%%
\begin{document}

\maketitle

1. Run the ARTS control file \texttt{rtcalc.arts} to calculate the spectrum of
atmospheric zenith opacity in the microwave spectral range for a
midlatitude-summer atmosphere over a wet land surface. You can use the
attributive plotting script \texttt{plot\_bt.py} to visualize the results.

\textbf{Questions}
\begin{itemize}
  \item To which species do these lines belong? (You can find this out by
    playing with the absorption species selection in the ARTS control file.)

    \begin{figure}[ht]
      \centering
      \begin{subfigure}[t]{0.49\textwidth}
        \includegraphics[width=\textwidth]{{plots/opacity_N2+H2O_0.0km_0deg}.pdf}
        \caption{\ce{H2O} and \ce{N2}}
      \end{subfigure}
      \hfill
      \begin{subfigure}[t]{0.49\textwidth}
        \includegraphics[width=\textwidth]{{plots/opacity_N2+O2_0.0km_0deg}.pdf}
        \caption{\ce{O2} and \ce{N2}}
      \end{subfigure}
      \caption{Zenith opacity separately for \ce{O2} and \ce{H2O} molecules.}
      \label{figure:abs_molucules}
    \end{figure}

  \clearpage
  \item We speak of window regions where the zenith opacity is below 1. Where
    are they?

    \begin{figure}[ht]
      \centering
        \includegraphics[width=\textwidth]{{plots/opacity_N2+O2+H2O_0.0km_0deg}.pdf}
      \caption{Zenith opacity.}
    \end{figure}
\end{itemize}

\clearpage

2. Brightness temperature is a unit for intensity. It is the temperature that a
blackbody should have to give the same intensity as measured. Mathematically,
the transformation between intensity in SI units and intensity in brightness
temperature is done with the Planck formula. Calculate and display the
atmospheric brightness temperature spectrum for different hypothetical sensors:

\begin{itemize}
  \item A ground-based sensor looking in the zenith direction.
  \item A sensor on an airplane (z = 10\,km) looking in the zenith direction.
\end{itemize}

\textbf{Questions}
\begin{itemize}
    \item In Plot (a), why do the lines near 60\,GHz and near 180\,GHz appear
        flat on top?
    \item In Plot (b), why is the line at 180\,GHz smaller than the line at
        120\,GHz, although its zenith opacity is higher?
    \item Describe the difference between plots (a) and (b). What happens to
        the lines, what happens to the background? Can you explain what you
        see?
\end{itemize}

% figures abs cross sections
\begin{figure}[h]
  \centering
  \begin{subfigure}[h!]{0.49\textwidth}
      \includegraphics[width=\textwidth]{{plots/bt_N2+O2+H2O_0.0km_0deg}.pdf}
      \caption{Ground based.}
    \end{subfigure}
    \hfill
    \begin{subfigure}[h!]{0.49\textwidth}
      \includegraphics[width=\textwidth]{{plots/bt_N2+O2+H2O_10.0km_0deg}.pdf}
      \caption{10\,km height.}
    \end{subfigure}
  \caption{Brightness temperature for a sensor positioned at the ground and at
    10\,km height.}
  \label{figure:abs_pressure}
\end{figure}

\clearpage

3. Make the same calculation for a satellite sensor (z = 800\,km) looking nadir
(straight down).

\textbf{Questions}
\begin{itemize}
  \item Explain the brightness temperature simulated in the window regions.
  \item Why does the line at 22\,GHz look different from the others?
  \item Explain the “funny” shape close to the center of the \ce{O2} line at
    120\,GHz (you may have to perform an ARTS simulation focused around that
    frequency).
\end{itemize}

\begin{figure}[h]
\centering
  \includegraphics[width=\textwidth]{{plots/bt_N2+O2+H2O_800.0km_180deg}.pdf}
  \caption{Brightness temperature seen from a sensor looking down.}
\end{figure}
\end{document}
